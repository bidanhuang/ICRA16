\section{Conclusion} 

In this paper a robotic system for sewing a stent graft is presented. We successfully teach the robot to sew with a curved needle as well as adapt to needle posture changes using a vision guidance system (needle posture re-detection). We analyse the robustness of our method quantitatively by two experiments: fabric piercing and needle regripping, which are the two critical parts for completing one sewing stitch. We show that our system is able to accomplish the sewing task effectively. Our vision system reconstructs the needle shape with sub-millimetres accuracy, and hence is able to guide robustly the robot to sew. The successful rate of our system is over 86$\%$. The failure case is due to the limited workspace caused by the robot joint limit. This problem can be leased by optimizing the task priority in the null space~\cite{yang2015}.

In this study, we use a single robot to manipulate the needle. The successful rate can be improved by using a second robot to cooperate the manipulation. This would make the tasks of both robot easier and also optimize each robot’s joint utilization. Our current system works as an open-loop manner for the needle adaptation part. Tracking the needle in real-time and implementing visual servoing algorithm is desired to increase the sewing accuracy. Further, the sewing presented is conducted in traditional way with needle drivers, in the future, customized sewing device will be designed by us to replace the needle holder to drive the needle performing fine movement. 

The system presented in this paper is a primary and promising study of robot hand sewing. Applications of the presented system and method are not limited only for stent graft sewing; it is also a promising technique for automating robotic suturing.

\section{Introduction}

%\begin{enumerate}
%\item{1} Bimanual manipulation-suturing is important and challenging
%\item{2} Our task is stent graft manufacturing
%%\item{3} Unsolved problems:Path planning,Adaptive to new context
%%\item{4} Our solutions:Vision + learn from demonstration,
%%Path planning - learning from demonstration, needle tracking, tool tracking,
%%Adaptive to new context - object centric approach + vision guided.
%\end{enumerate}
%~\cite{bidan2013grasp} 

A stent graft is a tubular structure composed of fabric supported by a metal mesh called a stent. It is widely used for a variety of conditions for endovascular intervention, but most commonly is used to reinforce an aneurysm.
Clinically, each stent graft needs to be customised to the patient anatomy, with fenestrations (openings) on the graft body to maintain the patency of important branches to vital organs. They often come at a significant cost in addition to long delays in manufacturing, largely due to the labour intensive manual tasks involved, subjecting patients to the risk of rupture during the waiting period and precluding treatment to patients presenting acutely. Improved manufacturing of personalised stentgrafts is therefore a critical unmet clinical demand and robot assisted manufacturing is being pursued.

This paper focus on the key process of the stentgraft manufacturing: sewing the stent to a fabric tube. The shape of the fabric tube is pre-designed for the patient anatomy and pre-manufactured. Unlike normal sewing, to sew the stent requires a ``3D sewing'' technology. Curved needles are commonly used for this task, as it can be pierced in and pierced out from one side of the fabric. We take a similar approach to robotize this task. 

Curved needle is widely used in surgery. Needle piercing is an important task for surgery robots. 
